\chapter{ 南通大学本科毕业设计(论文)学术不端行为处理办法(试行)}

为进一步提高本科毕业论文(设计)质量,加强学术道德和学风建设,营造诚信学术氛围,推动广大本科生科学引用文献资源,规范本科毕业设计(论文)管理,根据教育部《关于树立社会主义荣辱观进一步加强学术道德建设的意见》、《关于严肃处理高等学校学术不端行为的通知》、《学位论文作假行为处理办法》等有关文件精神和《南通大学学术不端行为处理规程(试行)》(通大〔2012〕32号)等文件规定,防范和处理本科毕业设计(论文)学术不端行为,特制订本办法。 

第一条 \quad 本办法适用于本科毕业设计(论文),本科生应严格遵守学术规范,恪守学术道德,弘扬优良学风,在指导教师指导下独立完成毕业设计(论文)。出现毕业设计(论文)学术不端行为的,学校将按本办法对当事人进行处理。 

第二条 \quad 本科毕业设计(论文)学术不端行为包括以下情形:
\begin{enumerate}
    \item 购买、出售毕业设计(论文)或组织毕业设计(论文)买卖的;由他人代做、为他人代做毕业设计(论文)或者组织毕业设计(论文)代做的。
    \item 使用他人已发表的数据、图表等内容未经授权或未注明出处的。 
    \item 照搬他人论文或著作中的实验结果及分析、系统设计和问题解决办法而没有注明出处或未说明借鉴来源的。
    \item 有过度引用行为的。引用他人内容已按规范注明出处,但文字复制比超过$30\%$的为抄袭,文字复制比大于$50\%$的为严重抄袭。
    \item 伪造数据的。 
    \item 有其他毕业设计(论文)学术不端行为的。
\end{enumerate}

第三条 \quad 学校将加强学术诚信建设,建立毕业设计(论文)学术诚信检查制度,每年在答辩前对本科毕业设计(论文)统一进行学术诚信审查和抄袭检测,检测结果作为毕业设计(论文)学术不端行为的基本认定依据。 

第四条 \quad 学术不端行为的认定和处理 
\begin{enumerate}
    \item 对举报或检查发现毕业设计(论文)有学术不端行为情形,学校将组织有关人员对其进行调查认定。经查实确认有上述学术不端行为的,由相关学院对论文作者进行批评教育、责令改正,并视情节轻重责令其修改论文、重新撰写论文、推迟答辩、取消学位(毕业)申请(答辩)资格等处理;已经获得学历学位的,依法撤销所获学位,注销所获学历证书。 
    \item 对于出现剽窃、抄袭等学术不端行为的学生,学校将依据《南通大学学生纪律处分规定》进行处理;对于为他人代做、出售毕业设计 (论文)或者组织毕业设计 (论文)买卖、代做的学生,按《南通大学学术不端行为处理规程(试行)》(通大〔2012〕32号)给予相应处理。 
    \item 对论文学术不端行为当事人做出处理决定前,学校将告知并听取当事人的陈述和申辩。当事人对处理决定如有异议的,可以在收到处理决定之日起5个工作日内向校学风建设领导小组提出书面申诉。对当事人提出的申诉,学校予以组织复查,对异议内容进行调查认定,并及时作出复查结论,告知申诉人。当事人对学校复查结果如有异议的,可以在接到复查决定之日起15个工作日内向省学风建设领导小组提出申诉。 
    \item 本科生毕业设计(论文)学术不端行为违反有关法律法规规定的,依法追究相关法律责任。 
\end{enumerate}

第五条 \quad 毕业设计(论文)指导教师应对所指导学生进行学术道德、学术规范教育,并对毕业设计(论文)研究和撰写过程予以指导、严格把关,从源头上防范、制止学术不端行为。对未履行学术道德和学术规范教育职责、指导工作不到位、把关不严或指使、放任作假行为,导致所指导的本科毕业设计(论文)存在学术不端行为的,将视情节轻重,追究该导师的相应责任。 

第六条 \quad 学校将毕业设计(论文)学术不端行为检查情况纳入二级教学单位教学状态评估与年度考核内容。对频繁或大面积出现毕业设计(论文) 学术不端行为或者学术不端行为影响恶劣的,学校将及时予以通报,并按规定追究相关责任。 

第七条 \quad 各学院可结合其学科、专业特点制定相关认定标准和实施细则,但对于抄袭的认定标准不得低于本办法所规定的标准。 

第八条 \quad 本办法自2014届毕业生开始实施,由教务处负责解释。 