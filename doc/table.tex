\chapter{表格}\label{cha:table}
引入样式宏包cnbasesetup.sty时,传入\verb|tab|参数引入表格绘制的宏包。
\begin{verbatim}
    \usepackage[super,tikz,tab,list,math]{style/cnbasesetup}
\end{verbatim}

\section{表格的定义环境}\label{sec:table-def-env}


\verb|tabular|及\verb|tabular*|环境是表格绘制的基本环境。tabular 环境可以用来排版带有横线和竖线的表格,\LaTeX 自动确定表格的宽度。tabular*环境与 tabular 环境类似,只是可以用参数指定表格的整体宽度,此外列参数必须在第一列后面的某个地方包含一个合适的表达式 (见下面说明)\footnote{在数学模式下使用的 array 环境的语法和参数的意义与 tabular 环境中的完全一样}。他们的语法格式如下:
\begin{verbatim}
    \begin{tabular}[位置可选参数]{列格式参数}
    每行由\\划分
    \end{tabular}
    \begin{tabular*}{宽度}[位置可选参数]{列格式参数}
    每行由\\划分
\end{tabular*}
\end{verbatim}



\section{表格环境参数格式}
\parwords{位置可选参数}
当然,有些时候我们需要更细致地控制表格的位置。\verb|table|环境中额外的位置控制参数为我们提供了一套控制表格位置的方法。下面我们给出一个简单的例子。
\begin{table}[!hpbt]
    \caption[表格插入位置控制选项]{\enspace 表格插入位置控制选项}
    \label{tab:table-position-options}
    \centering
    \normalsize% fontsize 
    \setlength{\tabcolsep}{4pt}% column separation
    \begin{tabular}{ll}
    \toprule
    选项 & 位置 \\
    \midrule
    h	& 将浮动元素的位置设定为 here\\
    t	& 将浮动元素的位置设定为页面的上方(top)\\
    b	& 将浮动元素的位置设定为页面的底部(bottom)\\
    p	& 将浮动元素仅放置在一个特殊的页面\\
    !	& 重新设置 \LaTeX{} 的一个内部参数\\
    H	& 浮动元素精确地放置在文本中所出现的位置\\
    \bottomrule
    \end{tabular}
\end{table}
其中, \verb|\begin{table}[h]|,方括号中的参数 \verb|h| 意味着 here,即大约插入到当前上下文位置,而不是完全精确地插入到给定位置。 参数 \verb|!| 决定了 \LaTeX{} 如何判断一个浮动元素的位置够不够“好”,参数 H 需要引入float包,它有可能会造成一些错误,有时候等价于h!。表~\ref{tab:table-position-options}~中列出了参数的可选值。总之,\verb|table|环境位置控制参数与 \verb|figure| 环境一致。

\parwords{列必选参数}
这个参数控制表格的列格式,故又称为列格式参数。在这个参数中,对每一列必须有一个相应的格式符号,另外还可能包含相应于表格左右边界和列间距的其它项。列格式符号可以取下列值:
\begin{table}[!hpbt]
    \caption[表格列格式参数]{\enspace 表格列格式参数}
    \label{tab:table-col-options}
    \centering
    \normalsize% fontsize 
    \setlength{\tabcolsep}{4pt}% column separation
    \begin{tabular}{ll}
    \toprule
    位置参数 & 位置 \\
    \midrule
    l	& 列中文本左对齐\\
    r	& 列中文本右对齐\\
    c	& 列中文本居中\\
    p	& \makecell[l]{宽度: 指定列的文本宽度,由宽度参数给出,\\ 列中文本按该宽度自动换行}\\
    |	& 画一条竖直线\\
    ||	& 画二条紧相邻的竖直线\\
    *\{num\}\{col-format\}& \makecell[l]{包含在col-format中的列格式被复制成num份,\\ 例如 *\{5\}\{|c\} 等价于 \texttt{|c|c|c|c|c}}\\
    \bottomrule
    \end{tabular}
\end{table}

\section{表格文本行中的命令}
表格中的每一水平行都由 \verb|\\| 结束。这些行由一组彼此之间用 \& 符号分开的列条目组
成。因此每一行应具有与列定义中列中相同数目的列条目,其中有些条目可以是空白的。
\begin{enumerate}
    \item \emph{ \backslash tabularnewline 命令}~~ \verb|\tabularnewline| 命令用于强制表格中一行的结束,而 \backslash\backslash 除了可以结束整个一行表格内容外,还可以在单个列的内容中实现换行。
    \item \emph{ \backslash hline 命令}~~ 这条命令只能位于第一行前面或紧接在行结束命令 \verb|\\| 的后面,表示在刚结束的那一行画一根水平的直线。如果这条命令位于表格的开头,那么就会在表格顶部画一横线,横线的宽度与表格的宽度相同。 放在一起的两条水平 \verb|\hline| 命令就会画出两条间隔很小的水平线。
    \item \emph{ \backslash cline\{n-m\}  命令}~~ \verb|\cline{n-m}| 命令的放置同 \verb|\hline| 命令,并且在一行中可以出现多次。该命令从第 n 列的左边开始,画一条到第 m 列右边结束的水平线。
    \item \emph{ \backslash vline 命令}~~ 该命令画一条竖直线,其高度等于其所在行的行高。用这种命令,可以得到那些不是贯穿整个表格的竖直线。
    \item \emph{ \backslash multicolumn\{列数\}\{ALIGNMENT\}\{CONTENT\} 命令}~~ 该命令用于合并列单元格,并可以在合并的单元格内进行内容的对齐,即该表格单元占据的列数,ALIGNMENT代表表格内容的对齐方式(填*,l,c或者r),CONTENT则是表格单元里的内容。例如,\backslash multicolumn\{3\}\{|c|\}\{内容\} 表示将内容合并到三列,并居中对齐。
    \item \emph{ \backslash multirow 命令}~~ 该命令用于合并行单元格,并可以在合并的单元格内进行内容的对齐,其基本语法为:
    \begin{verbatim} \multirow[⟨vpos⟩]{⟨nrows⟩}[⟨bigstruts⟩]{⟨width⟩}[⟨vmove⟩]{⟨text⟩} \end{verbatim}
    
    其中参数意义如下:
    \begin{itemize}
        \item \verb|⟨vpos⟩|~ 定义了合并后单元格内文本竖直位置的对齐方式,参数 \verb|t| 表示顶部对齐, \verb|b| 表底部部对齐,\verb|c|表示居中对齐,缺省默认值为 \verb|b|;
        \item \verb|⟨nrows⟩|~ 定义了合并后单元格纵向跨越的行数。该列的其他行必须留空,否则 \verb|\multirow| 创建的内容会覆盖已有数据。当 \verb|⟨nrows⟩| 为正值时,纵向范围涵盖当前行及下方 \verb|(⟨nrows⟩-1)| 行;为负值时则涵盖当前行及上方 \verb|(-⟨nrows⟩-1)| 行;\verb|⟨nrows⟩| 允许使用分数值,便于进行精细调整;
        \item \verb|⟨bigstruts⟩|~  参数主要在使用 bigstrut 宏包时生效。其数值表示跨行区域内所有 \verb|\bigstrut| 命令的使用次数总和。具体计数规则为:(1)无参数版本 \verb|\bigstrut| 计为 2 次,(2)带位置参数的 \verb|\bigstrut[⟨x⟩]|(\verb|⟨x⟩|为 \verb|t| 或 \verb|b|)计为 1 次,默认值为 0;\verb|⟨bigstruts⟩| 参数可添加字母前缀(\verb|t|、\verb|b| 或 \verb|tb|)实现精细控制(详细内容参见 \href{https://texdoc.org/serve/multirow.pdf/0}{multirow宏包文档});字母与数值间可保留空格,格式为 \verb|t 3| 或 \verb|tb 2|;
        \item \verb|⟨width⟩|~ 参数指定文本排版宽度,特殊取值包括:(1) \verb|*| 表示采用文本的自然宽度,(2) \verb|=| 表示采用当前列(即 \verb|\multirow| 条目所在列)的预设宽度;
        \item \verb|⟨vmove⟩|~ 是用于微调的长度值:文本将在原始位置基础上,根据该长度值上移(若 \verb|⟨vmove⟩| 为负值则下移);
        \item \verb|⟨text⟩|~ 为实际文本内容。根据宽度设定方式不同,排版规则如下:(1)如果显式指定宽度时,文本将置于指定宽度的 \verb|\parbox| 中,可通过 \\ 强制换行;(2)如果宽度设为 \verb|*| 时,文本以行内模式 (LR mode) 排版,需多行显示时,应在 ⟨text⟩ 内嵌套 tabular 或 array 环境;(3)如果宽度设为 \verb|=| 时,仅当 \verb|\multirow| 条目位于预定义宽度的列中生效(如:\verb|p{}| 列、tabularx 的 X 列、tabulary 的 \verb|L/C/R/J| 列),文本将置于该列预设宽度的 \verb|\parbox| 中,非适用场景使用 \verb|=| 将导致异常(通常表现为列宽过大)。
    \end{itemize}
    \item \emph{ at表达式命令}~~ \verb|@{}|表达式为每行的某两个相邻的列之间插入文本,同时自动去掉原来在这两列之间插入的空白。
    \begin{figure}[!hpbt]
        \begin{minipage}{0.5\textwidth}
        \begin{lstlisting}[language=tex]
        \begin{tabular}{r@{.}l}
        3   & 14159 \\
        16  & 2     \\
        123 & 456   \\
        \end{tabular}
        \end{lstlisting}%
        \end{minipage}
        \begin{minipage}{0.45\textwidth}
        \centering
        \begin{tabular}{r@{.}l}
        3   & 14159 \\
        16  & 2     \\
        123 & 456   \\
        \end{tabular}
        \end{minipage}
    \end{figure}

    对于\verb|@{}|,有下面几点需要说明:
    \begin{itemize}
        \item 如果我们需要继续使用空白,必须在 \verb|@{}| 表达式的文本参数中必须包含 \verb|\hspace{<wsep>}| 命令,或者使用 \verb|!{}| 替换;\verb|<wsep>| 为空白长度,可以是指定的长度(例如 \verb|4pt| )或者弹性长度 \verb|\fill|;
        \begin{figure}[!hpbt]
        \begin{minipage}{0.5\textwidth}
        \begin{lstlisting}[language=tex]
        \begin{tabular}{r@{.\hspace{\fill}}l}
        3   & 14159 \\
        16  & 2     \\
        123 & 456   \\
        \end{tabular}
        \end{lstlisting}%
        \end{minipage}
        \begin{minipage}{0.45\textwidth}
        \centering
        \begin{tabular}{r@{.\hspace{\fill}}l}
        3   & 14159 \\
        16  & 2     \\
        123 & 456   \\
        \end{tabular}
        \end{minipage}
        \end{figure}
        \item 如果希望某两个特定列之间的间隔与缺省的标准间隔不同,可以在表格环境的行参数中相应的位置上放上 \verb|@{\hsapce{<wsep>}}| 控制,此时该处列间间隔将变成 \verb|<wsep>|;
        \begin{figure}[!hpbt]
        \begin{minipage}{0.5\textwidth}
        \begin{lstlisting}[language=tex]
        \begin{tabular}{r@{.\hspace{\fill}}l}
        3   & 14159 \\
        16  & 2     \\
        123 & 456   \\
        \end{tabular}
        \end{lstlisting}%
        \end{minipage}
        \begin{minipage}{0.45\textwidth}
        \centering
        \begin{tabular}{rl}
        3   & 14159 \\
        16  @{.\hspace{\fill}}& 2     \\
        123 & 456   \\
        \end{tabular}
        \end{minipage}
        \end{figure}
        \item \verb|@{}| 表达式中可以使用 \verb|\extracolsep{<wsep>}| 控制,使后面所有列间间隔在原来标准间隔的基础上增加\verb|<wsep>|大小;
        \item 在 \verb|tabular*| 环境中。必须使用 \verb|@{\extracolsep\fill}| 命令,使得后面所有列间距可以伸展到预定义的表格宽度;
        \item 一个表格即使左右边界没有竖线或其他表征符号,相应的位置与后面 (前面) 的列之间也会插入等于标准列间隔一半的空白;如果不希望有这些空白,可以在行参数开始或结束处使用 \verb|@{}| 表达式。
    \end{itemize}
\end{enumerate}


\section{表格浮动体环境}
\texttt{table}环境用于将表格进行浮动处理,使其可以自动调整到合适的位置,并能够添加表格标题和标签。一个简单的表格环境格式如下:
\begin{verbatim}
\begin{table}[位置可选参数]
  \caption{T.}
  \label{table:1}%
    ... ...
\end{table}
\end{verbatim}
这里需要注意:
(1)常用的位置选项与 \verb|figure| 环境一致,如表 \ref{tab:table-position-options} ,包含: \verb|h| :在当前位置放置表格; \verb|t| :将表格放置在页面的顶部; \verb|b| :将表格放置在页面的底部;\verb|p|:单独放置表格到一页等。
(2)给定表标题 \verb|\caption{<短标题>}{<长标题>}| ,可放置在表上方或者下方。
(3)给表打标签 \verb|\label{<标签名称>}| ,表格标签名称一般使用“table:”作为前缀,以方便辨认和分类管理。注意:这个命令必须位于 \verb|\caption|之后。
(4)可以使用命令  \verb|\ref{<标签名称>}| 引用表格,也可以使用命令 \verb|\pageref{<标签名称>}| 定位其所在页码。


\texttt{table}环境用通常与\texttt{tabular}或\texttt{tabular*}搭配使用,\texttt{table}环境控制表格的位置、标题及标签,\texttt{tabular}或\texttt{tabular*}绘制表格。同时,为了使表格在页面上居中,还会利用 center 环境或 \backslash centering 命令:
\begin{verbatim}
\begin{table}[位置可选参数]
  \begin{center}%让表格居中
  \caption{T.}%resnet50 on imagenet1k
  \label{table:1}%
    \begin{tabular}{列格式参数}
    每行由\\划分
    \end{tabular}
  \end{center}
\end{table}
\begin{table}[位置选项]
  \begin{center}%让表格居中
  \caption{T.}%resnet50 on imagenet1k
  \label{table:1}%
    \begin{tabular*}{宽度}{列格式参数}
    每行由\\划分
    \end{tabular*}
  \end{center}
\end{table}
\end{verbatim}


\section{表格样例}
我们先看一个表格样例:
\begin{lstlisting}[language=tex]
    \begin{table}[!htbp]
        \bicaption{这是一个样表。}{This is a sample table.}
        \label{tab:sample}
        \centering
        \footnotesize% fontsize
        \setlength{\tabcolsep}{4pt}% column separation
        \renewcommand{\arraystretch}{1.2}%row space 
        \begin{tabular}{l|cccc>{\columncolor{blue}}cccc}
            \hline
            \rowcolor{gray} 行号 & \multicolumn{8}{c}{跨多列的标题}\\
            \hline
            \cellcolor{red} R1 & $1$ & $2$ & $3$ & $4$ & $5$ & $6$ & $7$ & $8$\\
            R2 & $1$ & $2$ & $3$ & $4$ & $5$ & $6$ & $7$ & $8$\\
            \cline{2-9}% partial hline from column i to column j
            \multirow{2}{l}{R3-4}& $1$ & $2$ & $3$ & $4$ & $5$ & $6$ & $7$ & $8$\\
             & $1$ & $2$ & $3$ & $4$ & $5$ & $6$ & $7$ & \textcolor{red}{$8$}\\
            \hline
        \end{tabular}
    \end{table}
\end{lstlisting}
其中
\begin{itemize}
    \item \texttt{[!htbp]}:表示表格的浮动位置,\texttt{!}表示必须出现在文档的当前位置,\texttt{h}表示在当前位置,\texttt{t}表示在文本顶部,\texttt{b}表示在文本底部,\texttt{p}表示在浮动页。
    \item \texttt{\textbackslash bicaption\{中文标题\}\{英文标题\}}命令用于设置表格的标题,第一个参数为中文标题,第二个参数为英文标题。这个命令的完整版格式为\texttt{\textbackslash bicaption \textbackslash [中文表目录显示标题 \textbackslash ]\{中文标题\}\textbackslash [英文表目录显示标题 \textbackslash ]\{英文标题\}}。
    \item \texttt{\textbackslash label\{标签名\}}命令用于设置表格的标签,标签名可以自定义。
    \item \texttt{\textbackslash centering}命令用于设置表格居中。
    \item \texttt{\textbackslash footnotesize}命令用于设置表格字体大小。
    \item \texttt{\textbackslash tabcolsep\{4pt\}}命令用于设置表格列间距。
    \item \texttt{\textbackslash renewcommand\{\textbackslash arraystretch\}\{1.2\}}命令用于设置表格行间距。
    \item \texttt{\textbackslash begin\{tabular\}\{l|cccc>\{\textbackslash columncolor\{blue\}\}cccc\}}命令用于设置表格单元格中内容对齐方式,\texttt{l}表示左对齐,\texttt{c}表示居中对齐,\texttt{r}表示右对齐, |表示列之间画竖线隔开。\texttt{>\{\textbackslash columncolor\{blue\}\}}表示列的颜色。
    \item \texttt{\textbackslash hline}命令用于设置表格的横线。
    \item \texttt{\textbackslash multicolumn\{8\}\{c\}\{跨多列的标题\}}命令用于设置表格的跨列单元格。
    \item \texttt{\textbackslash multirow\{2\}\{*\}\{Row3-4\}}命令用于设置表格的跨行单元格。
    \item \texttt{\textbackslash cline\{2-9\}}命令用于设置表格的部分横线。
    \item \texttt{\textbackslash \textbackslash}命令用于设置表格的换行。
    \item \texttt{\&}命令用于行内单元格分割。
    \item \texttt{\textbackslash rowcolor\{gray\}}命令用于设置表格的行颜色。
    \item \texttt{\textbackslash cellcolor\{gray\}}命令用于设置表格的单元格颜色。
    \item \texttt{\textbackslash textcolor\{red\}\{红色文字\}}命令用于设置表格的文字颜色。
\end{itemize}
请见表~\ref{tab:sample}(注意:使用命令\texttt{\textbackslash ref\{tab:sample\}}引用表格)。
\begin{table}[!htbp]
    \bicaption{这是一个样表。}{This is a sample table.}
    \label{tab:sample}
    \centering
    \footnotesize% fontsize
    \setlength{\tabcolsep}{4pt}% column separation
    \renewcommand{\arraystretch}{1.2}%row space 
    \begin{tabular}{l|cccc>{\columncolor{blue}}cccc}
        \hline
        \rowcolor{gray} 行号 & \multicolumn{8}{c}{跨多列的标题}\\
        \hline
        \cellcolor{red} R1 & $1$ & $2$ & $3$ & $4$ & $5$ & $6$ & $7$ & $8$\\
        R2 & $1$ & $2$ & $3$ & $4$ & $5$ & $6$ & $7$ & $8$\\
        \cline{2-9}% partial hline from column i to column j
        \multirow[c]{2}*{R3-4}& $1$ & $2$ & $3$ & $4$ & $5$ & $6$ & $7$ & $8$\\
        & $1$ & $2$ & $3$ & $4$ & $5$ & $6$ & $7$ & \textcolor{red}{$8$}\\
        \hline
    \end{tabular}
\end{table}

许多专业排版的书籍和期刊都使用简单的表格,它们在行上下有适当的间距,而且几乎从不使用垂直线。许多 \LaTeX{} 表格示例(包括这本)展示了垂直线的使用(使用“|”)和双线的使用(使用“||”),这些在专业出版物中被认为是不必要的,而且会分散注意力。 booktabs 包可以方便地为  \LaTeX{} 表格提供这种专业性,它的文档 也提供了关于构成“良好”表格的指南。

简而言之,该包使用 \verb|\toprule| 表示最上面的规则(或线), \verb|\midrule| 表示出现在表格中间的规则(例如标题下面的规则), \verb|\bottomrule| 表示最下面的规则。这确保了规则的宽度和间距是可以接受的。此外, \verb|\cmidrule| 可以用来表示跨越指定列的中间规则。

%以下示例对比了 booktabs 的使用和两种等效的正常 LaTeX 实现(第二个示例需要在序言中使用 \verb|\usepackage{array}| 或 \verb|\usepackage{dcolumn}|,第三个示例需要在序言中使用 \verb|\usepackage{booktabs}|)。

\begin{example}三线表是科技期刊与论文常用的表格,本例给出了一种简单的三线表格绘制样例。
\begin{lstlisting}[language=tex]
\begin{table}[!hbtp]
    \caption[峰谷电价]{\enspace 峰谷电价}
    \label{tab:elec-price}
    \centering
    \normalsize% fontsize 
    \setlength{\tabcolsep}{4pt}% column separation
    \begin{tabular}{m{40pt}<{\raggedright}m{120pt}<{\raggedright}m{80pt}<{\raggedright}}
    \toprule
    周期 & 时段 & 电价 (CNY/kWh) \\
    \midrule
    Peak & 12:00--15:00, 18:00--21:00 & 1.3782\\
    Flat & 9:00--10:00, 15:00--18:00, 21:00--23:00 & 0.8595\\
    Valley & 1:00--7:00, 23:00--0:00 & 0.3658\\
    \bottomrule
    \end{tabular}
\end{table}
\end{lstlisting}
其显示效果如表\ref{tab:elec-price}。
\end{example}
\begin{table}[!hbtp]
    \caption[峰谷电价]{\enspace 峰谷电价}
    \label{tab:elec-price}
    \centering
    \normalsize% fontsize 
    \setlength{\tabcolsep}{4pt}% column separation
    \begin{tabular}{m{40pt}<{\raggedright}m{120pt}<{\raggedright}m{80pt}<{\raggedright}}
    \toprule
    周期 & 时段 & 电价 (CNY/kWh) \\
    \midrule
    Peak & 12:00--15:00, 18:00--21:00 & 1.3782\\
    Flat & 9:00--10:00, 15:00--18:00, 21:00--23:00 & 0.8595\\
    Valley & 1:00--7:00, 23:00--0:00 & 0.3658\\
    \bottomrule
    \end{tabular}
\end{table}


\begin{example}在绘制表格时,常常需要对表格进行注释,然而 \verb|tabular|环境不支持脚注。这里给出一种基于 \verb|minipage| 环境的 \verb|tabular| 表格脚注解决方案,供读者参考。
\begin{lstlisting}[language=tex]
\begin{table}[tb]
    \begin{center}
	\begin{minipage}{0.6\textwidth}%这里需要根据实际表格做合适调整。
    \caption[正常组小鼠各主要组织的砷分布]{\enspace 正常组小鼠各主要组织的砷分布}
    \label{tab:mice-As}
    \normalsize% \zihao{5}
    \setlength{\tabcolsep}{4pt}% column separation
    \begin{tabular}{m{100pt}<{\raggedright}m{110pt}<{\raggedright}}
    \toprule
    脏器 & 砷含量 ($\bar{X} \pm S, \mu g/g$)\\
    \midrule
    大脑 &  BDL\\
    肺 &  $0.1023 \pm 0.0659$\\
    心脏 &  BDL\\
    \bottomrule
    \end{tabular}
    {\newline BDL:低于检出限}
\end{minipage}
\end{center}
\end{table}
\end{lstlisting}
其显示效果如表\ref{tab:mice-As}。
\end{example}
\begin{table}[tb]
    \begin{center}
	\begin{minipage}{0.6\textwidth}
    \caption[正常组小鼠各主要组织的砷分布]{\enspace 正常组小鼠各主要组织的砷分布}
    \label{tab:mice-As}
    \normalsize% \zihao{5}
    \setlength{\tabcolsep}{4pt}% column separation
    \begin{tabular}{m{100pt}<{\raggedright}m{110pt}<{\raggedright}}
    \toprule
    脏器 & 砷含量 ($\bar{X} \pm S, \mu g/g$)\\
    \midrule
    大脑 &  BDL\\
    肺 &  $0.1023 \pm 0.0659$\\
    心脏 &  BDL\\
    \bottomrule
    \end{tabular}
    {\newline BDL:低于检出限}
\end{minipage}
\end{center}
\end{table}
制作\LaTeX{}表格的的更多范例,请参见 \href{https://wikibooks.cn/wiki/LaTeX/Tables}{WiKibook Tables}。