\chapter{结论与展望}\label{chap:conclusion}
\section{论文总结}
本研究的目标是基于贝叶斯脑假设\cite{beal2003VB,Friston_freeEnergy_2010,daunizeau2010observingA,daunizeau2010observingB,Mathys2011HGF}研究动态环境中智能体的感知与学习。本文深入研究了智能体在动态环境中对不确定性的表征及高效学习与决策。研究受到认知神经科学中人类在动态环境下的决策行为及生物神经系统中神经元低阶交互的信息表征机制的启发,在贝叶斯脑假设下,从理论层面构建了不同类型动态环境的概率描述模型,并且智能体内部以一致的层次化方式抽象表征了不同类型动态环境的不确定性。本文中基于不同类型的动态环境构建了一系列的模型及仿真,详尽论述了所构建智能体对动态环境不确定性(波动)的高效感知与表征,并且是可概率描述的动态环境中最优的学习推断模型。这为动态环境下自适应行为的机制提供了一个有前景的最优的概率解释模型。本文主要开展了下述内容的研究:
\begin{enumerate}
	\item {\verb|基于多尺度不确定性表征的感知模型|}:针对一般动态环境中不确定性表征问题,提出了一般层次化布朗滤波,有效解决了动态环境中高斯假设下不确定性的多尺度表征与学习的问题。
	动态环境中的智能体必须具备对环境的不确定性理解与表征的机制。在高斯与布朗运动假设下,环境的时变不确定性被布朗运动的扩散强度矩阵所表征。进一步地,为了提高对环境不确定性表征的灵活性、高效性、时变动态性,扩散强度矩阵被另一个布朗运动参数化表征。以此类推,可构建多层级的布朗运动对环境不确定性(波动)进行多尺度的表征与学习。 
	进一步地,针对一般层次化布朗滤波的反演,基于变分贝叶斯方法整合平均场假设及基于一步牛顿法的高斯二次型近似,提出了一般层次化布朗滤波及其变体模型的反演方案,把推断的优化问题转变为状态的解析更新,简化了模型的逆向推断,提高了推断效率。从而,模型的变分逆被一组高效的、可解析的更新方程所实现。然后,我们在鱼的u-turn轨迹中测试了模型的性能。结果表明,模型很好的捕捉了轨迹的动态特征。
	\item {\verb|基于多元伯努利分布的动态环境不确定性表征与感知学习|}:动态地多元二值环境是神经科学中常用的实验环境。通常使用独立的多元伯努利分布描述多元二值的状态。一方面,如果智能体仅仅使用这个模型描述动态波动的环境,那么便丢失了对环境结构及不确定性的理解能力。另一方面,被伯努利分布所描述的二值环境的全部信息包含在自然参数向量里,这样智能体便可利用层次化布朗滤波跟踪自然参数向量即可。独立的伯努利分布的缺陷也会被低阶交互地自然参数所弥补。为了评估模型在动态波动的二值环境中的感知与决策能力,我们设想了一个双臂赌博任务,并依据贝叶斯决策理论构建了响应模型,与基于层次化的布朗滤波的感知模型共同形成贝叶斯智能体。仿真研究的结果表明,智能体很好的理解了动态波动的二值环境。几乎以最优的方式在二值环境中产生了决策行为。
	\item {\verb|基于指数分布的动态环境不确定性表征与感知学习|}:动态地多元指数分布环境是神经科学中常用的实验环境。通常使用独立的多元指数分布描述环境的状态。一方面,如果智能体仅仅使用这个模型描述动态波动的环境,那么便丢失了对环境结构及率强度参数不稳定波动的理解能力。另一方面,被指数分布所描述的动态环境的全部信息包含在率强度参数向量里,这样智能体便可利用层次化布朗滤波跟踪率强度参数向量的对数即可。独立的指数分布的缺陷也会被低阶交互地率强度参数所弥补。为了研究模型在动态指数分布信号环境中的感知,我们构建了一个动态率强度的仿真任务,模型很好的捕获了动态指数分布信号环境的基本统计特征。进一步地,我们消融了层次化布朗滤波中的波动层,获得了假设外部环境波动为常量的新模型。通过贝叶斯模型选择,消融模型不能很好的表征环境波动而变成次优模型。
	\item {\verb|基于分类分布的动态环境不确定性表征与感知学习|}:为了表征分类分布,我们把其状态固定到一个多元高斯分布空间中考虑,每个状态的概率与其在多元高斯分布内的密度成正比。这样我们便把离散状态的不确定性转换成了多元高斯分布的中心位置及其不确定性。进一步地,我们可以使用层次化布朗滤波跟踪上述多元高斯分布的中心位置及其不确定性。
	为了研究模型在动态分类分布环境中的感知,我们构建了一个动态分类感知的仿真任务,模型很好的捕获了动态分类分布环境的基本统计特征。进一步地,我们消融了层次化布朗滤波中的波动层,获得了假设外部环境波动为常量的新模型。通过贝叶斯模型选择,消融模型不能很好的表征环境波动而变成次优模型。
	\item {\verb|基于不确定性调节的自适应学习率机制|}:为了研究不确定性调节的自适应学习率,文中提出了基于高斯随机游走和伯努利分布的层次化贝叶斯模型,并且其学习率具有基于信念和环境不确定性所调节的自适应机制,能够有效且准确地推断动态二分类环境中的隐概率。在这个层次化贝叶斯模型中,隐状态的更新规则是具有恒定学习率的Rescorla-Wagner更新规则的扩展,即恒定学习率扩展为基于信念和环境不确定性调节的时变自适应学习率。
	作为模型自适应推断的重要机制,受信念和环境不确定性调控的自适应学习率有助于促进对生物自适应行为机制的理解,也为进一步开发新型自适应机器学习模型提供了理论框架。
	本文中基于随机生成的数据验证了学习率自适应机制的有效性和高效性。
\end{enumerate}


本文在贝叶斯脑假设下立足于动态环境中自适应学习模型的构建,结合了神经科学与机器学习,以探索动态环境下不确定性的表征、学习与决策。本项研究的重要任务之一就是探索一般动态环境下的不确定性(波动)的表征机制。这是保障智能体在动态环境下准确认知环境的基础,也是后续动态环境下决策的前提。本项研究的另一个重要任务就是抽象现实世界不同类型的动态环境,并构建相应动态环境下不确定性表征的贝叶斯模型。总之,本项研究内容新颖,具有较高的模型理论前瞻性和创新性,具有较好的应用前景和价值,其主要贡献和研究意义概括如下:
\begin{enumerate}
	\item {\verb|提出了一般层次化布朗滤波|}:受到贝叶斯脑假设下层次化先验信念的表达及神经元集群中成对低阶交互的信息表征机制的启发,基于高斯和布朗假设,构建了层次化的布朗运动以对动态环境不确定性(波动)形成层次化多尺度表征。有效地解决了动态环境中高斯假设下不确定性的多尺度表征与学习问题。
	
	\item {\verb|提出了层次化布朗滤波及其变体模型的解析反演方案|}: 针对一般层次化布朗滤波及其变体模型的反演,提出了一般层次化布朗滤波及其变体模型的变分反演方案, 整合了变分贝叶斯方法、平均场假设和基于一步牛顿法的高斯二次型近似,把较高计算开销的优化问题转变成了低计算开销的状态更新,简化了模型的逆向推断,提高了推断效率。
	
	\item {\verb|提出了动态指数族分布环境下不确定性表征与学习方案|}: 针对动态指数族分布所描述的环境中不确定性感知与学习,提出了基于一般层次化布朗滤波的贝叶斯感知模型,有效解决了指数族分布环境中充分统计量评估及环境不确定性认知的问题,揭示了动态指数族分布环境下不确定性感知的一般机制。
	
	\item {\verb|提出了一种基于不确定性的学习率调节机制|}: 主要针对动态环境下的学习机制问题,有效解决了智能体动态环境下学习率调节与形成问题,提高了智能体对环境的感知效率。总之,基于信念和环境不确定性的学习率自适应调节机制是模型信息处理的标志,也为进一步探讨智能体高效学习模型的构建提供了可借鉴的机制。
	
	
	%\item {\verb|提出了多元动态二值环境的概率描述及相应的层次化贝叶斯感知模型|}:受到社交认知学习实验的启发,我们构建了动态多元二值环境的概率描述。并基于层次化布朗滤波构建了层次化贝叶斯模型,以高效认知动态多元二值环境。同时,在变分贝叶斯和平均场近似下,我们获得了动态多元二值环境中伯努利自然参数和不确定性的高效更新规则。后验期望状态的学习过程被不确定性(波动)所调节。
	%\item {\verb|提出了多元动态指数分布环境的概率描述及相应的层次化贝叶斯感知模型|}:受到脑内信号肥尾特征及泊松过程事件等待时间的启发,我们构建了动态多元指数分布环境的概率描述。并基于层次化布朗滤波构建了层次化贝叶斯模型,以高效认知动态多元指数分布环境。同时,在变分贝叶斯和平均场近似下,我们获得了动态多元指数分布环境中指数分布率强度对数及其不确定性的高效更新规则。后验期望状态的学习过程被不确定性(波动)所调节,呈现出预测编码的更新方式。
	%\item {\verb|提出了多元动态分类分布环境的概率描述及相应的层次化贝叶斯感知模型|}:受到认知实验(如小鼠走迷宫,赢者通吃)的启发,我们构建了动态分类分布环境的概率描述。并基于层次化布朗滤波构建了层次化贝叶斯模型,以高效认知动态分类分布环境。同时,在变分贝叶斯和平均场近似下,我们获得了动态分类分布环境中离散高斯分布的位置及其不确定性的高效更新规则。后验期望状态的学习过程被不确定性(波动)所调节,呈现出预测编码的更新方式。
\end{enumerate}

\section{未来工作展望}
动态环境的认知建模对研究生物适应性行为及解析自适应性行为背后的机制至关重要。动态环境下,智能体需要高效的信息的处理机制,以应对复杂多变的动态环境。本文基于理论层面构建了动态环境下一般层次化布朗滤波。未来还有多个方向可以开展深入研究:
\begin{enumerate}
	\item {\verb|认知实验|}:动态环境的不确定性表征和认知学习是考察生物智能体学习和认知特征的很好环境。依照本文提供的模型进行有针对的实验设计,有望解析出人类在复杂动态环境下的行为机制。
	\item {\verb|神经网络模型|}:一般层次化布朗滤波为我们提供了一个很好的框架。基于这个框架我们可以构建更一般的神经网络模型。探索大脑执行贝叶斯推断的一般机制。
	\item {\verb|自适应信息处理|}: 本文提供了动态环境下不确定性的层次化表征,形成了自适应学习率更新规则。进一步解析动态学习率的调节机制,是下一步很好的研究方向。
\end{enumerate}