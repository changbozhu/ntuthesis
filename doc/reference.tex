\chapter{参考文献}\label{cha:referenceformat}

参考文献引用过程以实例进行介绍,假设需要引用名为"Document Preparation System"的文献,步骤如下:
\begin{enumerate} \small
\item 使用Google Scholar搜索Document Preparation System,在目标条目下点击Cite,展开后选择Import into BibTeX打开此文章的BibTeX索引信息,将它们copy添加到ref.bib文件中。这种方法比较快速,但是会存在不准确的问题。最安全的做法是直接到论文在线网页找到其官方提供的BibTeX索引信息。

\item 索引第一行 \verb|@article{daunizeau2010observingB},|中 \verb|daunizeau2010observingB| 即为此文献的标签(label) (\textbf{中文文献标签也必须使用英文label},一般遵照:姓氏拼音+年份+标题第一字拼音的格式),想要在论文中索引此文献,有两种索引类型:(1)\emph{文本类型引用格式}:使用命令\verb|\citet{daunizeau2010observingB}|引用文献\verb|daunizeau2010observingB|,显示格式如此处所示 \citet{daunizeau2010observingB}; (2)\emph{括号类型引用格式}:使用命令\verb|\citep{daunizeau2010observingB}|引用文献\verb|daunizeau2010observingB|,显示格式如此处所示 \citep{daunizeau2010observingB}。\emph{推荐采用括号类型引用格式}。
\end{enumerate}


\parwords{多文献索引用英文逗号隔开} 命令 \verb|\cite|、\verb|\citep|、\verb|\citet| 均支持同时引用多个参考文献。我们以 \verb|\cite| 为例,展示同时引用多个文献的方法,即命令 \verb|\cite{daunizeau2010observingB, Mathys2011HGF}|引用文献,显示格式如此处所示 \cite{daunizeau2010observingB, Mathys2011HGF}。

更多例子如:

\citet{Mathys2011HGF} 根据 \citet{daunizeau2010observingA,daunizeau2010observingB} 的研究,首次提出...。其中关于... \citep{adolphs2003cognitive, YAU2023107799},是当前认知...得到迅速发展的研究领域 \citep{NIPS2012_c399862d, goodfellow2016book}。引用同一著者在同一年份出版的多篇文献时,在出版年份之后用
英文小写字母区别,如:\citep{daunizeau2010observingA, daunizeau2010observingB} 和 \citet{daunizeau2010observingA, daunizeau2010observingB}。同一处引用多篇文献时,按出版年份由近及远依次标注。例如 \citep{beal2003VB,Friston_freeEnergy_2010, Mathys2011HGF, sutton2018reinforcement}。

使用著者-出版年制(authoryear)式参考文献样式时,中文文献必须在BibTeX索引信息的 \textbf{key} 域(请参考ref.bib文件)填写作者姓名的拼音,才能使得文献列表按照拼音排序。参考文献表中的条目(不排序号),先按语种分类排列,语种顺 序是:中文、日文、英文、俄文、其他文种。然后,中文按汉语拼音字母顺序排列,日文按第一著者的姓氏笔画排序,西文和 俄文按第一著者姓氏首字母顺序排列。如中 \citep{wei2021_bayesian,matThm_zhang2013}、英 \citep{AMARI1993185,mirza2018human}、俄 \citep{Dubrovin1906}。

如此,即完成了文献的索引,请查看下本文档的参考文献一章,看看是不是就是这么简单呢?是的,就是这么简单!

不同文献样式和引用样式,如著者-出版年制(authoryear)、顺序编码制(numbers)、上标顺序编码制(super)可在 \projectname.tex中对 cnbooksetup.sty 调用实现,参考文献处理器、引用风格的选择请参见章节\ref{ssec:cnbooksetup}。

若在上标顺序编码制(super)模式下,希望在特定位置将上标改为嵌入式标,可使用 \verb|\citetns|命令(如此处显示 \citetns{beal2003VB,Friston_freeEnergy_2010, Mathys2011HGF, sutton2018reinforcement}) 和 \verb|\citepns|命令(如此处显示 \citepns{AMARI1993185,mirza2018human})。



默认情况下,只有被实际引用的参考文献库(如ref.bib)中的文献会列在pdf文件参考文献列表内。命令 \verb|\nocite{*}| 可以显示参考文献库(ref.bib)中的所有参考文献(包括未引用文献)。