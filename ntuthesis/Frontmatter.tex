%---------------------------------------------------------------------------%
%->> Frontmatter
%-

%---------------------------------------------------------------------------------------------------------------
%-> 摘要
%-
%- 摘要内容要求在800~1200字,应简要说明本论文的目的、方法、结果和结论。
%- 要突出论文的创新之处。语言力求精炼、准确。在本页的最下方另起一行,
%- 注明本文的关键词(3-5个)。
%- 摘要内容要求在800~1200字,应简要说明本论文的目的、方法、结果和结论。
%- 要突出论文的创新之处。语言力求精炼、准确。在本页的最下方另起一行,
%- 注明本文的关键词(3-5个)。
%摘要是以提供文献内容梗概为目的,不加评论和补充解释,简明、确切地记述文献重要内容的短文。其要素一般包括:
% -(1)目的——研究、研制、调查等的前提、目的和任务,所涉及的主要范围;
% -(2)方法——所用的原理、理论、条件、对象、材料、工艺、结构、手段、装备、程序等;
% -(3)结果——实验的、研究的结果,数据,被确定的关系,观察结果,得到的效果,性能等;
% -(4)结论——结果的分析、研究、比较、评价、应用,提出的问题,今后的课题,假设,启发,建议,预测等; 
%
%写摘要时不得简单地重复题名中已有的信息,要排除在本学科领域中已成常识的内容,要用第三人称的写法。
%应采用“对……进行了研究”、“报告了……现状”、“进行了……调查”等记述方法,不使用“本文”、“作者”等作为主语。
%摘要的第一句不要与题目重复;取消或减少背景信息,只表示新情况、新内容;不说空洞的词句,
%如“本文所讨论的工作是对过去×××的一个极大地改进”、“本工作首次实现了……”、“经检索尚未发现与本文类似的工作”等;
%此外,作者的打算及未来的计划不能纳入摘要。
%
%列出3~5个关键词,以关键字与课题间关联性由强到弱顺序排列,关键词之间用分号相隔,结束处不用标点符号。
\begin{abstract}    
\parwords{目的}  近期,主动推理的机制框架被提出作为发展意识统一理论的原则性基础,有望解决该领域的概念分歧。我们认为,要实现这一愿景,当前基于主动推理框架的提案需进一步完善,以形成真正的意识过程理论。

\parwords{方法}  提升机制性理论的途径之一,是采用计算模型等形式化方法来实现、调适和验证提出的概念框架。本文系统考察了计算建模方法如何助力完善主动推断与意识关联的理论提案,重点关注:(1)这些模型在容纳不同意识维度和实验范式方面的开发广度与成效;(2)如何通过仿真与实证数据检验和改进模型。

\parwords{结果}  尽管当前研究已取得鼓舞人心的成果,但我们认为这些探索仍处于初级阶段。要提升模型的结构效度与预测效度,必须通过未观测的新神经数据进行实证检验。


\parwords{结论}  主动推断要成为完备的意识理论仍面临关键挑战:模型需能解释广泛的意识现象谱系,特别是涵盖经验的现象学特征。尽管存在这些不足,该方法已被证明是推动理论发展的有效路径,为未来研究提供了重要潜力。

\keywords{什么什么,细胞,相关性}
\end{abstract}

%英文摘要上方应有题目,内容与中文摘要相同。
%在英文题目下面第一行写研究生姓名,专业名称用括弧括起置于姓名之后,
%研究生姓名下面一行写导师姓名,格式为Directed by...。
%最下方一行为英文关键词(Keywords 3-5个)
\begin{abstract*}
\parwords*{Purpose}  The abstract serves both as a general introduction to the topic and as a brief, non-technical summary of the main results and their implications. The abstract must not include subheadings (unless expressly permitted in the journal's Instructions to Authors), equations or citations.  Most journals do not set a hard limit however authors are advised to check the author instructions for the journal they are submitting to.

\parwords*{Methods}  The abstract serves both as a general introduction to the topic and as a brief, non-technical summary of the main results and their implications. The abstract must not include subheadings (unless expressly permitted in the journal's Instructions to Authors), equations or citations. Most journals do not set a hard limit however authors are advised to check the author instructions for the journal they are submitting to.

\parwords*{Results}  The abstract serves both as a general introduction to the topic and as a brief, non-technical summary of the main results and their implications. The abstract must not include subheadings (unless expressly permitted in the journal's Instructions to Authors), equations or citations.  Most journals do not set a hard limit however authors are advised to check the author instructions for the journal they are submitting to.


\parwords*{Conclusion} The abstract serves both as a general introduction to the topic and as a brief, non-technical summary of the main results and their implications. The abstract must not include subheadings (unless expressly permitted in the journal's Instructions to Authors), equations or citations.  Most journals do not set a hard limit however authors are advised to check the author instructions for the journal they are submitting to.

\keywords*{Variational Bayesian method, Free energy principle, Cognitive neuroscience}
\end{abstract*}
