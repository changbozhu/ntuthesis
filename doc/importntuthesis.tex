\section{导入 \projectname.cls文档类及相关配置宏包}
\subsection{导入\projectname.cls文档类}
\begin{lstlisting}[language=tex]
    \documentclass[twoside,doctor,academic]{style/ntuthesis}
    %- Multiple optional arguments:
    %- [<academic|professional>]% 
    %- [<bachelor|master|doctor|postdoctoral>]%
    %- [<oneside|twoside>]% oneside print, twoside print, or without print
    %- [fontset=<adobe|none|...>]% specify font set instead of automatic detection
    %- [draft]% show draft version information
    %- [scheme=plain]% thesis writing of international students
    %- [standard options for ctex book class: draft|paper size|font size|...]%
 \end{lstlisting}%
在导入 \projectname.cls文档类时,提供了下述选项。
\begin{enumerate}
    \item \emph{指定文档类型}:在引入文档类时,可以从两选项"[<academic|professional>]"中选一个来指定学位论文的类型,即“学术型academic与“专业型professional”。对于本科毕业设计和博士后报告均选择设置"academic"即可。进一步地,需要通过选项"[<bachelor|master|doctor|postdoctoral>]"指定学位论文的申请学位级别,依次对应“本科、硕士、博士、博士后”。这两组选项配合可完全确定学位论文类型与学位级别,也就完全确定了学位论文文档。
    \item \emph{打印模式}: 在引入文档类时,可以从三选项“[<oneside|twoside|null>]”中选一个来指定文档的打印模式,null为默认的缺省选项,不需要填写。oneside和twoside依次对应单面打印和双面打印。oneside和twoside均没有给出时,采用等边距空白。
    \item \emph{平台字体族选项}:模板默认自动检测操作系统平台,根据操作系统平台选择字体族,在引入文档类时,不需要指定“fontset”选项。当然,也可以通过选项“[<fontset=<windows|mac|adobe|none|>]”强制指定操作系统平台类型以手动确定操作系统平台字体族。详见章节\ref{sec:fontset}。
    \item \emph{启用手稿模式}:在引入文档类时,填入选项“[<draft>]”,模版会启用draft模式编译文档,并在文档第二页打印“Draft Version(日期)”。
    \item \emph{其它选项}: 其它受ctex book class支持的选项,包括: "[<draft|paper size|font size|...>]"。
\end{enumerate}



\subsection{导入基础配置宏包cnbasesetup.sty}\label{ssec:cnbasesetup}
\begin{lstlisting}[language=tex]
    \usepackage[super,list,table]{style/cnbasesetup}% document settings
    %- usage: \usepackage[option1,option2,...,optionN]{style/cnbasesetup}
    %- Multiple optional arguments:
    %- [bibauto|bibtex|biber]% set bibliography processor and package
    %- [<numbers|super|authoryear|alpha>]% set citation and reference style
    %   - <numbers>: textual: Jones [1]; parenthetical: [1]
    %   - <super>: textual: Jones superscript [1]; parenthetical: superscript [1]
    %   - <authoryear>: textual: Jones (1995); parenthetical: (Jones, 1995)
    %
    %- [lscape]% provide landscape layout environment
    %- [color]% provide color support via xcolor package
    %- [tikz]% provide complex diagrams via tikz package
    %- [table]% provide complex tables via ctable package
    %- [list]% provide enhanced list environments for algorithm and coding
    %- [math]% enable some extra math packages
    %- [xlink]% disable link colors
\end{lstlisting}%
\begin{enumerate}
    \item \emph{指定参考文献处理器}:在引入cnbooksetup宏包时,我们可以通过从三个选项"[bibauto|bibtex|biber]"中选一个来指定学位论文参考文献处理器。
        \begin{enumerate}
            \footnotesize
            \item "bibtex"选项指定参考文献处理器为bibtex。当参考文献处理器选择为“bibtex”时,在需要的位置直接使用带参考文献库名参数的命令“\backslash \projectname bibliography\{bibname\}”打印参考文献列表,这里没有文献库扩展名“.bib”。
            \item “biber”选项指定参考文献处理器为“biblatex”。当参考文献处理器选择为“biber”时,可以在导言区添加命令"\backslash addbibresource\{bibname.bib\}"以指定参考文献库为"bibname.bib",这个命令需要带上扩展名“.bib”,然后在需要的位置直接使用不带参数的命令“\backslash \projectname bibliography”打印参考文献列表。
            \item “biberauto”为文档自定义的选项,参考文献处理器由 \projectname.cls文档类指定。这也是缺省默认的处理器模式, \projectname.cls默认配置为biblatex。
        \end{enumerate}
    没有明确指定参考文献库名的情况下,模板默认配置参考文献库名为“ref.bib”。
    对于需要打印多段参考文献列表的,只能选择“biber”或“bibauto”作为参考文献处理器,比如引入“综述”及“综述参考文献”。
    %
    \item \emph{配置参考文献引用风格}: 在引入cnbooksetup宏包时,可通过三个选项“[<numbers|super|authoryear>]”中选一个来指定参考文献引用风格。(1).number:按出现顺序排序,以数字方式索引引用;(2).super:按出现顺序排序,以数字方式索引引用,并且引用处采用上角标小数字引用;(3).authoryear: 以“作者 年份”方式索引引用。
    \item \emph{局部转置版面宏包}: 参数"[<lscap>]"指定文档引入宏包fancyhdr、pdflscape、textpos,提供了landscape环境。
    \item \emph{颜色宏包}: 参数"[<color>]"指定文档引入宏包xcolor,提供丰富的颜色支持。
    \item \emph{tikz绘图宏包}: 参数"[<tikz>]"指定文档引入额外扩展的绘图宏包tikz及自定义贝叶斯智能体绘图宏包“mlstyle/bayesianagent.sty”,提供绘图支持。
    \item \emph{table类宏包}: 参数"[<table>]"指定文档引入宏包ctable、multirow、longtable、makecell,提供绘图支持。
    \item \emph{list类宏包}: 参数"[<list>]"指定文档引入宏包:
        \begin{lstlisting}[language=tex]
            \RequirePackage{verbatim}% improve verbatim environment
            \RequirePackage{enumitem}% configure the enumerate environment
            \setlist[enumerate]{wide=\parindent}% only indent the first line
            \setlist[itemize]{wide=\parindent}% only indent the first line
            \setlist{nosep}% default text spacing
            \RequirePackage{listings}% source code
            \RequirePackage{algpseudocode,algorithm,algorithmicx}% algorithm
            \providecommand{\algname}{Algorithm}%
            \def\ALG@name{\algname}% rename label
        \end{lstlisting}%
    \item \emph{math宏包}: 参数"[<math>]"指定文档引入额外数学宏包“mathtools.sty”及自定义宏包“mlstyle/mlmath.sty”。
    \item \emph{禁用链接颜色}: 参数"[<xlink>]"指定文档禁用链接的颜色。
\end{enumerate}




\subsection{导入模版配置宏包ntuthesissetup.sty}
\begin{lstlisting}[language=tex]
        \usepackage{style/ntuthesissetup}% document settings
 \end{lstlisting}%



 
 \section{\projectname 学位论文信息配置命令}
 学位论文信息配置命令建议写入文件“\projectname /Frontinfo.tex”文件内。下面详细列出各个配置命令。
 \begin{enumerate}
    \item \emph{\backslash schoollogo\{\}}:配置学校logo。例如:\newline \backslash schoollogo[width=8.16cm,height=2.34cm]\{official/ntu/ntu-logo\}。
    \item \emph{\backslash schoolbadge\{\}}:配置学校校徽。例如:\newline \backslash schoolbadge[width=3.78cm,height=3.78cm]\{official/ntu/ntu-badge\}。
    \item \emph{\backslash degree\{\}}:指定学位类别,限填医学博士、工学博士、文学硕士、理学硕士、医学硕士、临床医学专业硕士等,例如:\backslash degree\{工学博士\}。
    \item \emph{\backslash classid\{\}}:中图分类号,例如:\backslash classid\{TP391.4\}。
    \item \emph{\backslash confidential\{yes|no\}\{period\}}:指定学位论文是否涉密yes|no。如果涉密,给出解密周期。对于不涉密论文,可以不进行配置,文档模版默认执行不涉密配置。如果涉密可按样例配置:\backslash confidential\{yes\}\{2\},其中解密周期为2年。
    \item \emph{\backslash title\{\}}:给定学位论文中文标题,例如:\backslash title\{南通大学研究生学位论文\}。
    \item \emph{\backslash title*\{\}}:给定学位论文中文标题,例如:\backslash title*\{Nantong University Graduate Thesis\}。
    \item \emph{\backslash author\{\}}:给定学位论文作者中文名,例如:\backslash author\{朱昌波\}。
    \item \emph{\backslash author*\{\}}:给定学位论文作者英文名,例如:\backslash author*\{Changbo Zhu\}。
    \item \emph{\backslash studentid\{\}}:学位论文作者学号,例如:\backslash studentid\{202100111002\}。
    \item \emph{\backslash advisor\{\}}:指导教师中文名,例如:\backslash advisor\{X**~教授\}。
    \item \emph{\backslash advisor*\{\}}:指导教师英文名,例如:\backslash advisor*\{Prof.~** X\}。
    \item \emph{\backslash coadvisor\{\}}:指导教师中文名,例如:\backslash coadvisor\{Y**~副教授\}。
    \item \emph{\backslash coadvisor*\{\}}:指导教师英文名,例如:\backslash coadvisor*\{Assoc. Prof.~** Y\}。
    \item \emph{\backslash major\{\}}:一级/二级学科中文专业名称,专业名称需要与学籍信息一致,例如:\backslash major\{模式识别与智能系统\}。
    \item \emph{\backslash major*\{\}}:一级/二级学科英文专业名称,专业名称需要与学籍信息一致,例如:\backslash major*\{Pattern Recognition and Intelligent Systems\}。
    \item \emph{\backslash majorcode\{\}}:一级/二级学科专业代码,专业名称需要与学籍信息一致,例如:\backslash majorcode\{081104\}。
    \item \emph{\backslash researchfield\{\}}:研究领域/研究方向,例如:\backslash researchfield\{认知神经计算\}。
    \item \emph{\backslash department\{\}}:培养单位/院系,例如:\backslash department\{电气与自动化学院\}。对于两个单位,可按下述格式书写:\backslash department\{电气工程与自动化学院 \backslash\backslash 神经网络与智能机器人研究所\}。
    \item \emph{\backslash completedate\{year\}\{month\}\{date\}}:指定完成日期,例如:\backslash completedate\{二〇二五\}\{十二\}\{二十五\}。
    \item \emph{\backslash foundation\{\}}:指定支持基金,例如:\backslash foundation\{××科学基金(30000000) \backslash\backslash ××科学研究基金(WKJ2008-0-000)\}。
\end{enumerate}