\chapter{南通大学学位论文格式要求}\label{appendix:ntuthesisrule}

研究生学位论文是研究生科学研究工作的全面总结,是描述其研究成果、代表其研究水平的重要学术文献资料,是申请和授予相应学位的基本依据。为规范我校博士、硕士学位论文形式,按照国家标准局颁布的《科学技术报告、学位论文和学术论文的编写格式》,结合我校实际情况,特制定本要求。

\section{学位论文字数要求}
研究生学位论文一般用中文撰写(特殊专业除外)。撰写学位论文应简明、扼要,既能够全面、真实反映个人的研究工作,达到相应申请学位水平,又不能抄袭或搬用别人的研究成果或理论(正常的引用除外,但需注明出处,且引用不宜篇幅过长)。学位论文字数可由学院根据学科特点以及专业(领域)相应的全国教育指导委员会要求自行规定,但不能低于学校规定的最低标准。学位论文字数(不包括图表)要求:
\begin{enumerate}
\item 自然科学类博士学位论文一般不少于5万字;
\item 自然科学类学术学位硕士学位论文一般不少于3万字,人文社科类学术学位硕士学位论文一般不少于4万字;
\item 专业学位硕士研究生学位论文字数要求由校专业学位研究生培养指导委员会根据各专业学位类别相应的全国教育指导委员会要求制定。
\end{enumerate}

\section{学位论文内容及编排顺序}
\begin{enumerate}
\item 封面
\item 学位论文原创性声明及使用授权声明
\item 扉页
\item 目录
\item 中英文摘要
\item 绪论
\item 正文
\item 结论或结语
\item 参考文献
\item 文献综述
\item 附录
\item 攻读学位期间取得的成果
\item 致谢
\end{enumerate}
如无法被上述部分囊括的,可自酌设定后,插入适当位置。

\section{学位论文各部分格式要求}
详见研究生院网站下载中心-\href{https://yjs.ntu.edu.cn/2024/1009/c7679a250968/page.htm}{\emph{南通大学博硕士学位论文格式模板(2024版)}}。

\section{论文印刷及装订要求}
\begin{enumerate}
\item 论文封面采用皮纹纸,全校统一格式。同时需要调整封面颜色为:\emph[crimson]{博士学位论文的封面为深红色},\emph{学术学位硕士学位论文封面为白色},\emph[light-green]{专业学位硕士论文封面为浅绿色}。
\item 双面打印,胶装。纸张方向:纵向;除封面外,所有页面边界为,上3.8厘米,下3.8厘米,左3.2厘米,右3.2厘米;页眉2.8厘米,页脚2.6厘米。对齐方式:两端对齐;首行缩进0.92厘米或2字符;行间距:固定值20磅。打印论文装订后的尺寸为285mm×205mm(版心尺寸为240mm×150mm)。
\end{enumerate}

\parwords{论文无附录者无需附录部分}
\parwords{论文无综述者无需综述及综述参考文献}
