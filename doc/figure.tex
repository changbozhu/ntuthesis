\chapter{图片插入}\label{cha:figure}

论文中图片的插入通常分为单图和多图,下面分别加以介绍:

单图插入:假设插入名为\verb|c06h06|(后缀可以为.jpg、.png、.pdf,下同)的图片,其效果如图~\ref{fig:c06h06}。
\begin{figure}[!htbp]
    \centering
    \includegraphics[width=0.40\textwidth]{docs/c06h06}
    \bicaption{Q判据等值面图,同时测试一下一个很长的标题,比如这真的是一个很长很长很长很长很长很长很长很长的标题。}{Isocontour of Q criteria, at the same time, this is to test a long title, for instance, this is a really very long very long very long very long very long title.}
    \label{fig:c06h06}
\end{figure}

如果插图的空白区域过大,以图片\verb|c06h06|为例,自动裁剪如图~\ref{fig:c06h06_trim}。
\begin{figure}[!htbp]
    \centering
    %trim option's parameter order: left bottom right top
    \includegraphics[trim = 60mm 80mm 60mm 60mm, clip, width=0.40\textwidth]{docs/c06h06}
    \bicaption[激波圆柱作用]{激波圆柱作用。}[Shock-cylinder interaction]{Shock-cylinder interaction.}
    \label{fig:c06h06_trim}
\end{figure}

多图的插入如图~\ref{fig:oaspl},多图不应在子图中给文本子标题,只要给序号,并在主标题中进行引用说明。
\begin{figure}[!htbp]
    \centering
    \begin{subfigure}[b]{0.35\textwidth}
      \includegraphics[width=\textwidth]{docs/oaspl_a}
      \caption{}
      \label{fig:oaspl_a}
    \end{subfigure}%
    ~% add desired spacing
    \begin{subfigure}[b]{0.35\textwidth}
      \includegraphics[width=\textwidth]{docs/oaspl_b}
      \caption{}
      \label{fig:oaspl_b}
    \end{subfigure}
    \\% line break
    \begin{subfigure}[b]{0.35\textwidth}
      \includegraphics[width=\textwidth]{docs/oaspl_c}
      \caption{}
      \label{fig:oaspl_c}
    \end{subfigure}%
    ~% add desired spacing
    \begin{subfigure}[b]{0.35\textwidth}
      \includegraphics[width=\textwidth]{docs/oaspl_d}
      \caption{}
      \label{fig:oaspl_d}
    \end{subfigure}
    \bicaption[总声压级]{总声压级。}[OASPL]{OASPL.}
    \figurenote{(a) 这是子图说明信息,(b) 这是子图说明信息,(c) 这是子图说明信息,(d) 这是子图说明信息。}
    \label{fig:oaspl}
\end{figure}